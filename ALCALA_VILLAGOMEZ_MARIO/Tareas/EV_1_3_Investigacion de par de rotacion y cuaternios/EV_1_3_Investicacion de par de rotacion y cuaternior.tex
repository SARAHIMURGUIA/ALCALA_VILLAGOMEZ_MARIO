\documentclass[letter,12pt]{report}
\usepackage[utf8]{inputenc}
\usepackage[spanish]{babel}
\usepackage{graphicx}
\usepackage{amsmath}
\usepackage{amsfonts}
\usepackage{amssymb}
\author{Mario Alcala}
\title{Rotaci\'on y Cuaternios}

\begin{document}
\maketitle

 \textbf{La rotacion y cuaternios} se divide dos partes:
 
Primero se discuten los resultados sobresalientes con rotaciones en tres
dimensiones, especificamente el teorema de Euler, este se demostra utilizando algebra lineal, y la formula de Rodrigues. En la formula de Rodrigues, se discute su derivacion ademas de la relacion que tiene con matrices de rotacion ya 
conocidas en la teora de rotaciones.

En la segunda parte tomamos la relacion entre rotaciones en el
espacio tridimensional y los cuaternios. En esta parte se revisa la relacion en entre la formula de Rodrigues y las rotaciones mediante cuaternios.\\

\textbf{Euler, Rodriguez y Rotacion R\emph{3}}\\


\textbf{Euler} conocido como la identidad de los cuatro cuadrados, dice que el producto de dos
numeros, cada uno es una suma de cuatro cuadrados, y es una
suma de cuatro cuadrados.

El teorema de \textbf{Euler}.Nos asegura que toda rotacion por un cierto angulo en cualquier espacio deja una linea recta que es el eje de rotacion.
(\textbf{Euler}) Si \textbf{R} es una matriz que representa una rotacion en \textbf{R}\textit{3}, entonces \textbf{R} tiene un vector propio \textbf{n} \textit{E} \textbf{R}\textit{3} tal que
\textbf{Rn} = \textbf{n},

\includegraphics[width=5cm]{../../Documents/7A/Cinematica de robots/cuatermios/Captura.JPG}

Figura captura: Derivacion de la matriz de rotacion en \textbf{R}\textit{3} alrededor de un eje unitario \textbf{n} por un angulo.\\

La formula de Rodrigues.
"Sea v un vector cualquiera en R3. Si \textbf{v}rot es el vector que se obtiene al rotar el
vector \textbf{v} alrededor del vector unitario \textbf{n} 2 \textbf{R}\textit{3} por un angulo" , la formula
de Rodrigues nos da las coordenadas de este vector:\\

\begin{center}
vrot = cos v + sen  (n x v) + (1 -cos a) \<n,v>n,
\end{center}

\includegraphics[width=5cm]{../../Documents/7A/Cinematica de robots/cuatermios/Rodriguez.JPG} 
En el sistema O'UVW está trasladado un vector p(6,-3,8) con respeto del sistema OXYZ. Calcular las coordenadas (rx , ry ,rz) del vector r cuyas coordenadas con respecto al sistema O'UVW son ruvw(-2,7,3) 8\\
\includegraphics[width=5cm]{../../Documents/7A/Cinematica de robots/cuatermios/matriz.JPG} \\
\includegraphics[width=5cm]{../../Documents/7A/Cinematica de robots/cuatermios/Plano.JPG} 



\end{document}